%\documentclass[twoside]{pwrthesis}
\documentclass[twoside]{iisthesis}
% ---
\usepackage{polski}
\usepackage[utf8]{inputenc}
\usepackage{amsmath}
\usepackage{tocloft}
\usepackage{listings}
\usepackage{algorithm}
\usepackage{algorithmic}
\usepackage{subcaption}
\usepackage{mathtools}
\usepackage{graphicx}
\usepackage[colorinlistoftodos]{todonotes}
\usepackage{url}
\usepackage{pgfplots, pgfplotstable}
\selectlanguage{polish}
% Dodane przeze mnie d
\usepackage{fancyvrb} % dla srodowiska Verbatim
\usepackage{color}
\usepackage{lscape}
\hypersetup{
    colorlinks,
    linkcolor={black!50!black},
    citecolor={black!50!black},
    urlcolor={black!80!black}
}

\definecolor{gray}{rgb}{0.4,0.4,0.4}
\definecolor{darkblue}{rgb}{0.0,0.0,0.6}
\definecolor{cyan}{rgb}{0.0,0.6,0.6}

\lstset{
  basicstyle=\ttfamily,
  columns=fullflexible,
  showstringspaces=false,
  commentstyle=\color{gray}\upshape
}

\lstdefinelanguage{XML}
{
  morestring=[b]",
  morestring=[s]{>}{<},
  morecomment=[s]{<?}{?>},
  stringstyle=\color{black},
  identifierstyle=\color{darkblue},
  keywordstyle=\color{cyan},
  morekeywords={xmlns,version,type}% list your attributes here
}

\lstset{
  language=XML,
   literate={ć}{{\'c}}1
}
\renewcommand*{\lstlistingname}{Kod źródłowy}
% definicje kolorow
\definecolor{ciemnoSzary}{rgb}{0.15,0.15,0.15}
\definecolor{szary}{rgb}{0.5,0.5,0.5}
\definecolor{jasnoSzary}{rgb}{0.2,0.2,0.2}

% Konfiguracja verbatima
\fvset{
	frame=single,
	numbers=left,
	fontsize=\footnotesize,
	numbersep=12pt,
%	framerule=.5mm,
	rulecolor=\color{ciemnoSzary},
%	fillcolor=\color{jasnoSzary},
	framesep=4pt,
	stepnumber=1,
	numberblanklines=false,
	tabsize=2,
%	formatcom=\color{szary}
}
\newcommand{\listequationsname}{Spis wzorów}
\newcommand{\equationcaption}[1]{\begin{flushright}\emph{#1}\end{flushright}}
\newcommand{\rightcaption}[1]{\begin{flushright}\emph{#1}\end{flushright}}
\newlistof{myequations}{equ}{\listequationsname}
\newcommand{\myequations}[1]{%
\addcontentsline{equ}{myequations}{\protect\numberline{\theequation}#1}\par}

\newcommand{\listofmyalgorithmsname}{Spis algorytmów}
\newlistof{myalgorithm}{algo}{\listofmyalgorithmsname}
\newcommand{\myalgorithm}[1]{%
\addcontentsline{algo}{myalgorithm}{\protect\numberline{\thealgorithm}#1}\par}


\newcommand{\listofmyfiguresname}{Spis rysunków}
\newlistof{myfigure}{figu}{\listofmyfiguresname}
\newcommand{\myfigure}[1]{%
\addcontentsline{figu}{myfigure}{\protect\numberline{\thefigure}#1}\par}

\floatname{algorithm}{Algorytm}

\newtheorem{mydef}{Definicja}



\begin{document}


\newcommand{\resultChart}[7][140]{
\def\dataS{{#2}}
	\begin{figure}[H]
	
\centering

\begin{center}
\begin{tikzpicture}
 
\begin{axis}[
ybar,
bar width=20,
legend style={at={(0.5,-0.25)},
anchor=north,legend columns=-1},
ylabel={Wartość miary},
symbolic x coords={\dataS},
xtick=data,
height=  {#1},
width=0.8\textwidth,
ymin=0, ytick={0,0.5,1},
ymax=1.5,
nodes near coords,
nodes near coords align={vertical},
]
\addplot coordinates { (\dataS,{#3}) };
\addplot coordinates {(\dataS,{#4}) };
\addplot coordinates { (\dataS,{#5}) };
\legend{Recall,Precission,F1-Score}
\end{axis}
\end{tikzpicture}
\end{center}
\caption{{#6}}
\myfigure{{#6}}
\label{{#7}}
\end{figure}
}


\pgfkeys{/pgf/number format/use comma}
\pgfkeys{/pgf/number format/.cd, set thousands separator={}}%
\nocite{*}
\title{ TITLE }
\titleEN{ TITLE EN}
\shortTitle{SHORT TITLE}
\author{Katatzyna Biernat }
\advisor{dr inż. Bernadetta Maleszka}
\instituteLogo{logos/pwr}
\slowaKluczowe{KEYWORDS}

\date{\number\the\year}

% Wstawienie abstractu pracy
	%\input {abstract}

\abstractSH{SHORT ABSTRACT}

\abstractPL{
ABSTRACT PL
}
\abstractEN{
ABSTRACT EN
}

\maketitle
\textpages


\graphicspath{ {img/} }
\DeclareGraphicsExtensions{.pdf,.png,.jpg}

 \chapter{Cel pracy}
 
	Celem pracy jest zaproponowanie i zbudowanie hybrydowego algorytmu rekomendacji. Składowymi docelowego algorytmu są metody kolaboratywnego filtrowania oraz metody filtrowania z analizą treści.  
 
 \chapter{Wstęp}
	 Wraz z rozwojem Internetu zmienił się sposób dostępu do informacji. Kiedyś to użytkownik musiał walczyć pozyskanie wiedzy; dzisiaj to informacje walczą u uwagę użytkowników. W świecie zalanym wiadomościami koniecznym wydaje się być zastosowanie filtra, który odsieje interesującą  i wartościową zawartość od tej niechcianej. Tak też z pomocą przychodzą zautomatyzowane mechanizmy rekomendacji.
	 
	 Jednakże sam koncept rekomendacji nie jest niczym nowym. Co więcej, zjawisko to możemy zaobserwować w naturze -- na przykład wśród mrówek, które podążają wyznaczoną (rekomendowaną) ścieżką feromonową w poszukiwaniu pożywienia.
	 
	 Ludzie od niepamiętnych czasów posiłkowali się opiniami innych aby ułatwić sobie dokonanie wyboru, od najbliższego grona znajomych do ekspertów i autorytetów.
	 
	 Wraz z rozwojem nauk informatycznych problem rekomendacji stał się problemem interesującym badaczy. Za pierwszy system rekomendacji uznaje się \textit{Tapestry} stworzony w laboratoriach Xerox Palo Alto Research Center w 1992 roku. Motywacją było odfiltrowanie rosnącej liczby niechcianej poczty elektronicznej \cite{id:FromTapestryToSVD}.
	 
	 Wkrótce później idea ta została rozszerzona przez takich graczy jak Amazon, Google, Pandora, Netflix, Youtube, Yahoo etc. aż do formy, jaką znamy dzisiaj: systemu, który sugeruje użytkownikom produkty, filmy, muzykę, strony internetowe na podstawie ich aktywności w sieci \cite{id:EvolutionOfRecommenderSystems}. 
	 
	 Wielkie koncerny internetowe stale poprawiają jakość swoich algorytmów rekomendacji. Najlepszym przykładem jest tutaj Netflix, który w październiku 2006 zorganizował ogólnodostępny konkurs na najlepszy algorytm. Zadaniem uczestników było ulepszenie algorytmu Cinematch. Już po siedmiu dniach od ogłoszenia konkursu trzy zespoły zdołały przebić Cinematch o 1.06\% \cite{id:NetflixPrize}\cite{id:NetflixPrizeRankings}.
	 
	 Systemy rekomendacji ulepszane są nieustannie, o czym świadczy chociażby organizowana rokrocznie konferencja\textit{ ACM International Conference on Recommender Systems}. Tematyka ta poruszana jest także na konferencjach \textit{European Conference on Information Retrieval}, \textit{European Conference on Machine Learning and Principles and Practice of Knowledge Discovery in Databases} i wielu innych. Mimo dużego stopnia zaawansowania wciąż istnieje pole manewru do ulepszania algorytmów rekomendacji i co za tym idzie zwiększanie zadowolenia użytkowników, które z kolei prowadzi do osiągania korzyści biznesowych.
	 
 
 \chapter{Przegląd istniejących rozwiązań}
	 \section{Problem rekomendacji}
	 \section{Podejście w oparciu o aktywność użytkownika}
	 \section{Podejście z wykorzystaniem bazy użytkowników}
 
 \chapter{Model systemu}
 
 \chapter{Algorytmy}
	 \section{Filtrowanie kolaboratywne}
		 \subsection{Matrix Factorization}
		 \subsection{Biased Matrix Factorization}
		 \subsection{SVD++}
	 \section{Filtrowanie z analizą zawartości}
		 \subsection{Konstrukcja sieci neuronowej}
		 \subsection{Uczenie sieci neuronowej}
	 \section{Algorytymy hybrydowe}
	 \section{Analiza złożoności i poprawności}
 
\chapter{Ocena eksperymentalna}
	\section{Opis metody badawczej}
	\section{Środowisko symulacyjne}
	\section{Metodologia}
	\section{Przeprowadzone eksperymenty}
	
\chapter{Wnioski}

\chapter{CHAPTER 1}
\section{SECTION}

\def\alghoritm1{Alghoritm 1}
\begin{algorithm}
\caption{\alghoritm1}
\myalgorithm{\alghoritm1}
\label{aq:algStat}
\begin{algorithmic}
\STATE $T \leftarrow \text{text under analysis}$
\FOR{each word $w \in T$}
    \STATE $S_{w}\leftarrow FIND\_SENTIMENT(w) $
    \IF {$S_{w}=POSITIVE$}
        \STATE $Sentiment[POSITIVE]++$
    \ELSIF{$S_{w}=NEGATIVE$}
        \STATE $Sentiment[NEGATIVE]++$
    \ELSE 
        \STATE $Sentiment[NEUTRAL]++$
    \ENDIF
\ENDFOR
%\STATE $x\in\{POSITIVE,NEGATIVE,NEUTRAL\}$
\RETURN $\arg\max_x Sentiment[x]$
\end{algorithmic}
\end{algorithm}


\def\schema1{Schema 1}
\begin{figure}[ht]
\caption{\schema1}
\myfigure{\schema1}
\label{fig:kdb}
\begin{center}
    <GRAPHIC>
\end{center}
\end{figure}

\section{Section 2}

\subsection{Subsection 1}

\subsubsection{Subsubsection 1}

\begin{mydef}
\textbf{Definicja} - pierwsza
\end{mydef}



 \clearpage
\appendix
\chapter{Appendix 1}


\clearpage
\pagestyle{plain}
\listofmyfigure
\listofmyequations
\listofmyalgorithm
\clearpage

%\bibliographystyle{apalike}%Used BibTeX style is unsrt

\bibliographystyle{iisthesis}
\bibliography{bibliography}

\end{document}

